% Chapter Template

\chapter{Conclusions} % Main chapter title

\label{Chapter7} % Change X to a consecutive number; for referencing this chapter elsewhere, use \ref{ChapterX}

\lhead{Chapter 7. \emph{Conclusions}} % Change X to a consecutive number; this is for the header on each page - perhaps a shortened title

%----------------------------------------------------------------------------------------
%	SECTION 1
%----------------------------------------------------------------------------------------

\section{Conclusions}
 
 
 We explained the evaluation for the performance of optimized Hadoop schedulers in various placement cases in chapter \ref{Chapter5}. The expectation is to increase the performance of Hadoop by speculative tasks execution, the evaluated results in chapter \ref{Chapter5} show that this assumption is not hold always. We shown that, the selection of appropriate numbers of reducers plays important role on performance of Hadoop. The evaluation results explained in chapter \ref{Chapter5} in respect to effects of number of reducers on Hadoop schedulers performance shown that the selection of reducers from 25\% to 50\% of number of mappers leads to best performance of Hadoop. The comparison results between experiment one and two proofs that increasing the number of reducers from single reducer to ten reducers improves the performance of Hadoop for all the schedulers and placement cases. As increasing the number of reducers provides diminishing returns to improve the the performance of Hadoop, the number of reducers should be well tuned for a given workload and resources, if possible.\\
 
 
 The evaluation results in chapter \ref{Chapter5} proofs that capacity share scheduler of Hadoop has better performance for jobs such as sorting data using terasort or any other similar datasets than fairshare scheduler for job completion time. The analysis of speculative task executions for capacity share show that it has marginally improvement for results in experiment two and three where appropriate number of reducers were selected. The marginally improvement for "cpt" case is because the job sizes were small and very few task were speculated, probably for large size jobs the speculation for cpt may lead to much better results. The analysis of collocation for capacity share scheduler explained in chapter \ref{Chapter5} shows that , while capacity share scheduler is more robust to collocation of datanodes in comparison to fairshare scheduler, it has better performance for collocated datanodes from the same cluster case in comparison to collocated datanodes from different clusters. \\      
 
 The evaluation results explained in chapter \ref{Chapter5} show that, fairshare scheduler of Hadoop has better fairness in all the cases in comparison to capacity share scheduler, but it has poor performance than capacity share scheduler. Fairshare scheduler performance is more sensitive to collocation of datanodes case, it's performance is decreased more in comparison to capacity share scheduler for collocation cases. To compare both collocation cases, fairshare scheduler has better performance for collocated datanodes from the same cluster in comparison to collocation of datanodes from different clusters. Fairshare scheduler utilizes more CPU in comparison to capacity share scheduler, probably this is because the complexity of algorithm used in scheduler design which requires more CPU cycles to process the jobs and guarantee the fairness. The speculative task execution does not lead to better performance of Hadoop using fairshare scheduler. The fairshare scheduler needs more CPU to process the jobs and by speculative task execution this demand goes higher which could lead to full utilization of resources, where no other non speculative  tasks can be executed and as result speculative task execution causes performance degradation.   \\
 
 Our evaluation results in chapter \ref{Chapter5} show that, the collocation of VMs acting as datanodes on a single physical machine degrades the performance  of Hadoop. Collocation of VMs‌ from same cluster has better results in comparison to collocation of VMs from different Hadoop clusters. This is because, in collocating both datanodes from the same Hadoop cluster, the namenode is aware of the status of the datanodes and assign them the task according to their process status. In collocated datanodes from two different Hadoop clusters, the namenode of every datanode is not aware about the status of the other datanode, so the decision of namenode asking to process data on collocated datanodes harms the performance other collocated datanode.  In collocation cases the CPU utilization is high(some time fully utilized during experiment) in comparison to baseline case and not receivable enough CPU cycles for task on execution time is possibly one of the reasons of performance degradation for all Hadoop schedulers.\\
 
 
 
   The analysis of current ICT infrastructure in Afghanistan explained in chapter \ref{Chapter6} provides the idea that, considering the current situation of Afghanistan where ICT infrastructure is very poor, security and lack of experts for the IT‌ fields are challenges, running the services on cloud operators have also the problem of unstable Internet. The projects such as NFO‌ brings the hope to have stable infrastructure of ICT in near future, on which organization can rely and use cloud services, but there are number of security and political challenges to the completion of the ICT‌ projects in Afghanistan.  

\section{Future Work}

 We addressed the evaluation and analysis of of Hadoop performance for set of schedulers using Terasort workload. We tuned number of reducers to see the impact of change on performance of Hadoop. There are possible extension to this analysis, where one can run similar experiments with different workloads and job sizes and validate the performance evaluation for all the cases considered in this thesis (or different ones). Also, it is possible to tune any other parameter of Hadoop such as queue configuration of schedulers, process of various job sizes, resource partitioning, etc, to analyse the results for performance of optimized Hadoop schedulers. The effect of collocation can be analysed for the cases where Hadoop nodes are collocated with other application in the data-centre. The test of similar experiment on bare-bone physical machines are another research field, because All the experiments are executed on VMs, which possibly may not provide the same results for execution on top bare-metal physical machines.\\
 
 The finding in this thesis shows that overall capacity share scheduler has better performance for job completion time in comparison to fairshare scheduler. The scheduler features can be improved by adding futures such as dynamic proportional scheduling in Hadoop \cite{dynamic} where  dynamically the priority of task execution is changed among organizations and the organizations that pays more dynamically receives more computational resources of shared Hadoop cluster. It may also be possible to add the feature of LATE scheduler where job completion time is calculated for all jobs, and the task which will complete in furthest future is speculated on fast nodes. Sailfish; \cite{sailfish} which is a MapReduce frame work can be used to optimize the performance of Hadoop by processing the output of map function that is consumed later by reducers which can improve up to 20\% the performance of Hadoop.  \\



