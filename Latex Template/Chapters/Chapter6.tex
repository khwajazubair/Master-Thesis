% Chapter Template

\chapter{Cloud Computing in Afghanistan} % Main chapter title

\label{Chapter6} % Change X to a consecutive number; for referencing this chapter elsewhere, use \ref{ChapterX}

\lhead{Chapter 6. \emph{Cloud Computing in Afghanistan}} % Change X to a consecutive number; this is for the header on each page - perhaps a shortened title

%----------------------------------------------------------------------------------------
%	SECTION 1
%----------------------------------------------------------------------------------------
Afghanistan is land lock country which most parts are enclosed by mountains. In the last two decades (80s and 90s), socio-political upheavals and war not only destroyed Afghanistan’s infrastructures and wealth, but it also destroyed the Information and Communication Technology (ICT) systems. Large  number of Afghan people scattered in and out of Afghanistan during last three decades of war.  This made the ICT services and facilities an essential issue for Afghanistan’s immediate reconstruction. After the Temporary Government in 2001 established in Afghanistan, the country obtained new view of political and socio-economic rehabilitation and structure. 

\section{Afghanistan and ICT}
 
  By establishment of the new government after 2011, Afghan government started to encourage the private sector to do investment in Afghanistan's ICT projects. Parallel to the investment of private sectors on ICT field  the government also started to develop ICT infrastructure and services in Afghanistan. As result of encourages big investors like Afghan Wireless Company (AWCC), Roshan, Areeba and Etisalat entered to ICT market of Afghanistan. In order to develop ICT and to fulfil the human resource requirements of the ICT field in Afghanistan the Afghan government Established Computer Science Faculties in many universities of Afghanistan, ICTI institute , creation of ICT Law and ICT strategy plan. Almost 80 percent of Afghan people have access to telecommunication that indicates rapid growth of ICT and implementation of ICT projects. In this section our focus is more on national projects which are offered or going to be offered by Afghan government.\cite{afg1}\\

Afghan people need to strengthen and diversify their economies, educate and engage their young people, develop the infrastructures that support economic growth, and lure back the educated professionals and business-people who have fled to other countries. ICT will be instrumental in meeting these challenges, but recent history shows that Afghanistan is suspicious of, and resistant to, technological change. Based on a report by the United Nations Science and Technology Group for Development (UNSTD), ICT strategies are often developed and publicized mainly to attract external investment to construct new infrastructures or to market hardware and software without giving sufficient attention to local concerns and requirements \cite{afg2}.\\


\subsection{National Optical Fiber Backbone}
This project will install a national backbone network across the country, which will support all the other projects (digital lines, microwave, etc). It will allow a high volume of national and international traffic and will connect major provinces and to neighbouring countries. This Network will link many of the principle cities of Afghanistan following the route of the national roadway system. This project is for the turnkey construction and operation of the complete Optical Fiber Communication (OFC) ring around Afghanistan which is estimated approximately 3.200 km length. In addition to linking many of the key cities. This project also calls for the construction of accesses to the backbone for the other cities and provinces not directly on the backbone route. Those access routes for the other cities and provinces will be taken up separately. There will be other major spurs off this main loop connecting to neighbouring countries Iran, Pakistan, Uzbekistan, Tajikistan, and Turkmenistan.\cite{afg1} 	


\section{Challenges}
Although, Afghan ICT had good progress in past years, but still there are challenges that needs time to be solved. Low levels of education and literacy, poor IT infrastructures, lack of ICT skilled human resources, lack of political interest to ICT can be counted as number of challenges. Security is a big challenge that not only ICT project implementation but over all Afghan government suffer from.  Lack of skilled human resources is another challenge, even use of the Internet requires a fairly complex set of skills and technology, at very least, one must have electricity, a communications line, a terminal capable of interacting across the communications lines, and (in most cases) a reasonable fluency in English \cite{afg3}. 


\section{Cloud Computing In Afghanistan}

The telecommunication companies currently launched 3G services, through which Afghan civilians can access Internet via cell phones. Considering that more than 80\% of the Afghan population has access to telecommunication services, launch of General Radio Packet Services (GPRS) and 3G services are major achievement for provision of Internet access to the civilians. For the organizations invested in ICT market of Afghanistan, poor IT infrastructure infrastructure, lack of security, required energy resources, IT experts all these challenges makes it difficult to establish and  maintain large data centres of their own for storage and process of their data. Still most of these companies use microwave of satellite links to connect to the Internet and very few and limited fibre links are available across the country as backbone links.\\

 The organization only need Internet connection with high enough speed to access the cloud services/infrastructure and use it according to their need. Using cloud services, the organizations are free of headache for maintenance, physical resources, security and hiring experts to manage IT resources. Though some of major ICT companies that have stable internet connection such as Roshan company rely on partial cloud service, but overall the poor infrastructure of ICT causes unstable Internet connections, which makes it difficult to even go for cloud services in Afghanistan. After completion of national projects in Afghanistan, such OFC‌ project  the organizations may have stable Internet connection that will ease the use and cultivation of cloud services among organization. The organizations may rely on cloud computing services where all the infrastructure, resources, storage, process and maintenance is provided by cloud operator to the clients via an Internet connection. \\
 
   
    
 


